\chapter{Appetizers and Sides}
%Some of these items can be made into meals in their own right.

\recipe[Homemade cheese crackers -- what's not to love?]{Cheddar Cheese Coins*}
\serves{10-12}
\preptime{30 minutes}
\cooktime{1 hour 50 minutes}
\vegetarian
\freezerfriendly
\begin{ingreds}
  8 oz extra-sharp cheddar cheese, shredded
  1 \nicefrac{1}{2} cups all-purpose flour
  1 tbsp cornstarch
  \nicefrac{1}{2} tsp salt
  \nicefrac{1}{4} tsp cayenne pepper
  \nicefrac{1}{4} tsp paprika
  8 tbsp unsalted butter, cut into 8 pieces and chilled
  3 tbsp water
\end{ingreds}

\begin{method}[Adjust oven rack to upper middle and preheat to \ftemp{350}.]
  Process cheese, flour, cornstarch, salt, cayenne, and paprika
  in food processor until combined, about 30 seconds.
  Scatter butter pieces over top and process until
  mixture resembles wet sand, about 20 seconds.
  Add water and process until dough forms a ball, about 10 seconds.
  Transfer dough to counter and divide in half.
  Roll each half into a 10-inch log, wrap in plastic wrap,
  and refrigerate until firm, at least 1 hour.

  Line 2 rimmed baking sheets with parchment paper.
  Unwrap logs and slice into \nicefrac{1}{4}-inch-thick coins,
  giving dough a quarter turn after each slice
  to keep the log round.
  Place coins 2 inches apart on the prepared baking sheets.

  Bake until light golden around edges, 22 to 28 minutes,
  switching and rotating sheets halfway through baking.
  Let coins cool completely on sheets before serving.
\end{method}


\recipe[These Filipino spring rolls can be made with a variety of meats, but we prefer to make them with shrimp, as below.]{Fried Lumpia}
\serves{8}
\preptime{}
\cooktime{}
\freezerfriendly
\begin{ingreds}
  1 tsp vegetable oil
  3 cloves garlic, minced
  \nicefrac{1}{2} cup onions, chopped
  1 lb small shrimp (or ground meat)
  1 tsp salt
  1 tsp black pepper
  1 tsp garlic powder
  1 tsp soy sauce
  3 cups cabbage, shredded
  2 cups carrots, shredded
  1 package lumpia wrappers
  egg wash
  oil to fry
\end{ingreds}

\begin{method}
  Saut\'e minced garlic and onions in vegetable oil.
  Add meat and heat until fully cooked.

  Stir in cabbage, carrots, salt, pepper, garlic powder,
  and soy sauce.
  Cook until cabbage just begins to cook down.
  Drain and allow to cool.

  Spoon filling onto wrappers,
  then roll and seal with egg wash.

  When ready to fry, heat frying oil in a pan.
  Deep fry rolls until well-browned and crispy.
  Serve hot.
\end{method}
\begin{tips}
  The rolled lumpia can be kept frozen
  until ready to fry.
\end{tips}

\recipe[]{Mushroom, Celery, and Wild Rice Pilaf*}
\serves{4}
\preptime{10 minutes}
\cooktime{20 minutes}
\vegan
\begin{ingreds}
  1 tbsp olive oil
  2 tsp minced garlic
  2 cups sliced mushroom
  2 celery stalks, chopped
  1 \nicefrac{1}{2} cups cooked quinoa
  \nicefrac{1}{2} cups cooked wild rice
  2 scallions, white and green parts, chopped
  salt and pepper
\end{ingreds}

\begin{method}
  Place a large skillet over medium-high heat
  and add the olive oil.

  Saut\'e the garlic until softened,
  about 2 minutes.

  Stir in the mushrooms and celry and saut\'e
  until the vegetables are lightly caramelized and tender,
  about 10 minutes.

  Stir in the quinoa and wild rice and saut\'e
  until warmed through, about 6 minutes.

  Season with salt and pepper and serve
  topped with scallions.
\end{method}
