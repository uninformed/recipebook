\documentclass[twoside, draft]{article}
\usepackage{comment}
\usepackage{units}
\usepackage{fancyhdr}
\usepackage{multicol}
\usepackage[%
    %a5paper,
    papersize={5.5in,8.5in},
    margin=0.75in,
    top=0.75in,
    bottom=0.75in,
    %twoside
    ]{geometry}

\usepackage{ifdraft}
\usepackage{makeidx}


\usepackage{xcolor}
\usepackage{graphicx}
\usepackage{tcolorbox}

\raggedcolumns
\setlength{\multicolsep}{0pt}
\setlength{\columnseprule}{1pt}

\makeatletter

\newif\if@mainmatter \@mainmattertrue

%% Borrowed from book.cls
\newcommand\frontmatter{%
    \cleardoublepage
  \@mainmatterfalse
  \pagenumbering{roman}}
\newcommand\mainmatter{%
    \cleardoublepage
  \@mainmattertrue
  \pagenumbering{arabic}}
\makeatother

%% Vary the colors at will

\definecolor{vegcolor}{rgb}{0,0.5,0.2}
\definecolor{frzcolor}{rgb}{0,0,1}
\definecolor{vgncolor}{rgb}{0.5, 0, 0.5}

% Your "recipes.sty" package starts here:
%% Thanks to alephzero for the excellent start:

\newcommand{\recipe}[2][]{%
    \newpage
    \lhead{}%
    \chead{}%
    \rhead{}%
    \lfoot{}%
    \rfoot{}%
    \subsection{#2}%
    \label{sec:#2}% in case I want to \pageref in the section opener
    \index{#2}
    \if###1##%
    \else
        \begin{center}
            \parbox{0.75\linewidth}{\raggedright\itshape#1}%
        \end{center}
    \fi
}
\newcommand{\serves}[2][Serves]{%
    \chead{#1 #2}}
\newcommand{\vegetarian}{%
    \rhead{\large\color{vegcolor}\textbf{V}}}

%% Optional arguments for alternate names for these:
\newcommand{\preptime}[2][Prep time]{%
    \lfoot{#1: #2}%
}
\newcommand{\cooktime}[2][Cook time]{%
    \rfoot{#1: #2}%
}


%% Optional argument is the width of the graphic, default = 1in
\newcommand{\showit}[3][1in]{%
    \begin{center}
        \bigskip
            \includegraphics[width=#1]{#2}%
            \par
            \medskip
            \emph{#3}
            \par
        \end{center}%
    }

%% Optional argument for a  heading within the ingredients section
\newcommand{\ingredients}[1][]{%
    \if###1##%
        {\color{red}\Large\textbf{Ingredients}}%
    \else
        \emph{#1}%
    \fi
}

%% Use \obeylines to minimize markup
\newenvironment{ingreds}{%
    \parindent0pt
    \noindent
    \ingredients
    \par
    \smallskip
    \begin{multicols}{2}
    \leftskip1em
    \rightskip0pt plus 3em
    \parskip=0.25em
    \obeylines
    \everypar={\hangindent2em}
}{%
    \end{multicols}%
    \medskip
}

\newcounter{stepnum}

%% Optional argument for an italicized pre-step
%% Also use obeylines to minimize markup here as well
\newenvironment{method}[1][]{%
    \setcounter{stepnum}{0}
    \noindent
    {\color{red}\Large\textbf{Instructions}}%
    \par
    \smallskip
    \if###1##%
    \else
        \noindent
        \emph{#1}
        \par
    \fi
    \begingroup
    \parindent0pt
    \parskip0.25em
        \leftskip2em
    \everypar={\llap{\stepcounter{stepnum}\hbox to2em{\thestepnum.\hfill}}}
}{%
    \par
    \endgroup}

\pagestyle{fancy}

%% ----- My additions -----
% temperature formatting
\newcommand{\ctemp}[1]{%
    $#1^\circ$C}
\newcommand{\ftemp}[1]{%
    $#1^\circ$F}

% new tags
\newcommand{\vegan}{%
    \rhead{\large\color{vegcolor}\textbf{V}\color{black}/\color{vgncolor}\textbf{V}}}
\newcommand{\freezerfriendly}{%
    \lhead{\large\color{frzcolor}\textbf{FF}}}

%
\makeatletter
\def\emptypage@emptypage{%
    \thispagestyle{plain}
    \hbox{}%
     \vspace*{\fill}
     \begin{center}
     \ifdraft{This page would be intentionally left blank\\ if I didn't need to know it was.}{}
     \end{center}
     \vspace{\fill}
     \newpage%
}%
% this version of cleardoublepage also properly clears the page style
\def\cleardoublepage{%
        \clearpage%
        \if@twoside%
            \ifodd\c@page%
                % do nothing
            \else%
                \emptypage@emptypage%
            \fi%
        \fi%
    }%
\makeatother

\newcommand{\chapter}[1]{
    %uses the \cleardoublepage above
    \cleardoublepage
    \rfoot{}
    \lfoot{}
    \chead{}
    \lhead{}
    \rhead{\textsc{#1}}
    \section{#1}
}

\newtcolorbox{tips}{title=\textit{Recipe tips}}

% End of "recipes.sty"

\usepackage{lmodern}
\usepackage{hyperref}
\usepackage{textcomp}
% MUST load ed, then enumitem, then draftwatermark. Why? Who knows.
% actually, I know. ed imports paralist, which can't be loaded after enumitem
\usepackage{ed}
\usepackage{enumitem}
\ifdraft{\usepackage[firstpage]{draftwatermark}}{}
\makeindex

\begin{document}
\setcounter{secnumdepth}{0}
\frontmatter
\title{Recipes}% really need a better title
\author{Compiled by T\&A}
\maketitle
\bigskip
\begin{center}
1st Edition, Preview 8.2
\end{center}
\newpage

% IDEA possibly move this section after the TOC
\chapter{About this book}
This book was lovingly typeset using \LaTeX~and
a modified version of the layout found at
\url{https://tex.stackexchange.com/a/366262}.

\smallskip
Several recipes have tags in the page headers.
{\color{frzcolor}FF} stands for freezer-friendly,
meaning the item can be frozen indefinitely 
(as far as we know) and thawed later.
These are often good choices for bulk cooking.
{\color{vegcolor}V} and {\color{vgncolor}V} denote
vegetarian and vegan recipes, respectively.

\smallskip
A quick word on release editions:
if you're holding a book that says
``Edition $x$, Preview $y$'' on the title page,
then I don't know how you got it,
but it probably wasn't one of us.
Preview editions are considered 
interim digital versions only 
and may contain incomplete recipes, misspellings, 
or other such oddities.

\smallskip
With that, enjoy the book, and happy cooking!

\bigskip ---T\&A

\newpage\rhead{}\tableofcontents

\mainmatter

\chapter{Sauces and Seasonings}

\recipe[]{Taco Seasoning}
\begin{ingreds}
  1 tbsp chili powder
  1 \nicefrac{1}{2} tsp ground cumin
  1 tsp salt
  1 tsp black pepper
  \nicefrac{1}{2} tsp paprika
  \nicefrac{1}{4} tsp garlic powder
  \nicefrac{1}{4} tsp onion powder
  \nicefrac{1}{4} tsp crushed red pepper
  \nicefrac{1}{4} tsp dried oregano
\end{ingreds}

\begin{method}
Mix all ingredients in a small bowl until well mixed.
\end{method}

% need that barbecue sauce recipe

\chapter{Appetizers and Sides}
%Some of these items can be made into meals in their own right.

\recipe[Homemade cheese crackers -- what's not to love?]{Cheddar Cheese Coins*}
\serves{10-12}
\preptime{30 minutes}
\cooktime{1 hour 50 minutes}
\vegetarian
\freezerfriendly
\begin{ingreds}
  8 oz extra-sharp cheddar cheese, shredded
  1 \nicefrac{1}{2} cups all-purpose flour
  1 tbsp cornstarch
  \nicefrac{1}{2} tsp salt
  \nicefrac{1}{4} tsp cayenne pepper
  \nicefrac{1}{4} tsp paprika
  8 tbsp unsalted butter, cut into 8 pieces and chilled
  3 tbsp water
\end{ingreds}

\begin{method}[Adjust oven rack to upper middle and preheat to \ftemp{350}.]
  Process cheese, flour, cornstarch, salt, cayenne, and paprika
  in food processor until combined, about 30 seconds.
  Scatter butter pieces over top and process until
  mixture resembles wet sand, about 20 seconds.
  Add water and process until dough forms a ball, about 10 seconds.
  Transfer dough to counter and divide in half.
  Roll each half into a 10-inch log, wrap in plastic wrap,
  and refrigerate until firm, at least 1 hour.

  Line 2 rimmed baking sheets with parchment paper.
  Unwrap logs and slice into \nicefrac{1}{4}-inch-thick coins,
  giving dough a quarter turn after each slice
  to keep the log round.
  Place coins 2 inches apart on the prepared baking sheets.

  Bake until light golden around edges, 22 to 28 minutes,
  switching and rotating sheets halfway through baking.
  Let coins cool completely on sheets before serving.
\end{method}


\recipe[These Filipino spring rolls can be made with a variety of meats, but we prefer to make them with shrimp, as below.]{Fried Lumpia}
\serves{8}
\preptime{}
\cooktime{}
\freezerfriendly
\begin{ingreds}
  1 tsp vegetable oil
  3 cloves garlic, minced
  \nicefrac{1}{2} cup onions, chopped
  1 lb small shrimp (or ground meat)
  1 tsp salt
  1 tsp black pepper
  1 tsp garlic powder
  1 tsp soy sauce
  3 cups cabbage, shredded
  2 cups carrots, shredded
  1 package lumpia wrappers
  egg wash
  oil to fry
\end{ingreds}

\begin{method}
  Saut\'e minced garlic and onions in vegetable oil.
  Add meat and heat until fully cooked.

  Stir in cabbage, carrots, salt, pepper, garlic powder,
  and soy sauce.
  Cook until cabbage just begins to cook down.
  Drain and allow to cool.

  Spoon filling onto wrappers,
  then roll and seal with egg wash.

  When ready to fry, heat frying oil in a pan.
  Deep fry rolls until well-browned and crispy.
  Serve hot.
\end{method}
\begin{tips}
  The rolled lumpia can be kept frozen
  until ready to fry.
\end{tips}

\recipe[]{Mushroom, Celery, and Wild Rice Pilaf*}
\serves{4}
\preptime{10 minutes}
\cooktime{20 minutes}
\vegan
\begin{ingreds}
  1 tbsp olive oil
  2 tsp minced garlic
  2 cups sliced mushroom
  2 celery stalks, chopped
  1 \nicefrac{1}{2} cups cooked quinoa
  \nicefrac{1}{2} cups cooked wild rice
  2 scallions, white and green parts, chopped
  salt and pepper
\end{ingreds}

\begin{method}
  Place a large skillet over medium-high heat
  and add the olive oil.
  
  Saut\'e the garlic until softened, 
  about 2 minutes.
  
  Stir in the mushrooms and celry and saut\'e
  until the vegetables are lightly caramelized and tender,
  about 10 minutes.
  
  Stir in the quinoa and wild rice and saut\'e
  until warmed through, about 6 minutes.
  
  Season with salt and pepper and serve 
  topped with scallions.
\end{method}

%%
\chapter{Meat-free Mains}
\begin{comment} % TODO finish chapter intro
We love meats, but meats are expensive,
sometimes making up as much as half the total
cost of our weekly grocery trip.
Vegetarian options can be as much about practicality
as about lifestyle choices.
\end{comment}

\recipe[So tasty you won't even notice it has no meat.]{Black Beans and Quinoa}
\vegan
\serves{4}
\preptime{10 minutes}
\cooktime{30 minutes}
\begin{ingreds}
  1 tbsp olive oil
  1 medium sweet or yellow onion, diced
  2 cloves garlic, minced
  \nicefrac{3}{4} cup uncooked quinoa, rinsed
  1 (15 oz) can black beans, drained and rinsed
  1 tsp chili powder
  1 tsp cumin
  \nicefrac{1}{4} tsp crushed red pepper flakes
  \nicefrac{1}{2} tsp black pepper
  salt to taste
  1 (10 oz) can original Rotel
  \nicefrac{1}{2} cup fresh cilantro, chopped
  1 \nicefrac{3}{4} cup vegetable broth
\end{ingreds}

\begin{method}
  In a large skillet, heat olive oil on medium-low heat
  and saut\'e diced onions until tender, about 4 minutes.
  Add garlic and saut\'e one additional minute.
  Add the remaining ingredients in the order listed above.

  Cover and bring to a boil.
  Reduce heat to a low boil and cook 15-20 minutes
  or until liquid is absorbed.
  Remove from heat and allow to sit covered for 5 minutes before serving.

  Fluff quinoa with a large spoon and serve.
\end{method}
\begin{tips}
  Don't forget to rinse the quinoa!
  Dry quinoa has a bitter coating that needs
  to be removed before cooking.
\end{tips}

%%
\chapter{Beef and Pork Mains}
\begin{comment} % chatper intro
  Naturally, all of these are delicious.
  Need I say more?
  % say something about when to use lean meats, I guess
\end{comment}

\recipe[]{Beef-Stuffed Bell Peppers}
\serves{4}
\preptime{15 minutes}
\cooktime{40 minutes}
\freezerfriendly
\begin{ingreds}
  4 red bell peppers, seeded with tops and membranes removed
  1 tbsp olive oil, plus extra for greasing the peppers
  1 lb extra-lean ground beef
  1 sweet onion, chopped
  1 cup mushrooms, chopped
  2 tsp minced garlic
  1 cup shredded spinach
  1 cup cooked quinoa
  salt and pepper
\end{ingreds}
\begin{method}[Preheat oven to \ftemp{350}.]
  Rub pepper with olive oil and place them
  hollow side down in baking pan.

  Bake the peppers for 10 minutes, until slightly softened,
  and remove them from the oven.
  Flip them hollow side up and set aside.

  Heat the olive oil in a large skillet over
  medium-high heat.

  Add the ground beef and saut\'e for about 10 minutes,
  until it is cooked through.
  Stir in the onion, mushrooms, and garlic and saut\'e
  for about 6 minutes, until translucent.

  Remove the skillet from the heat and stir in
  the spinach and quinoa.
  Season the filling with salt and pepper.

  Spoon the filling into the red peppers
  and return the baking dish to the oven.
  Bake 15 minutes and serve.
\end{method}
\begin{tips}
  Peppers can be packed in freezer bags and frozen
  after filling.
  Just thaw them in the refrigerator
  and bake for 20-25 minutes at \ftemp{350}.
\end{tips}

\recipe[Excellent served on taco boats or in a bowl with tortilla chips on the side.]{Chili}
\serves{8}
\freezerfriendly
\preptime{15 minutes}
\cooktime{2.5 hours}
\begin{ingreds}
  1 \nicefrac{1}{2} lb ground beef
  2 garlic buds, minced
  1 large onion, chopped
  1 tbsp butter
  1 tsp ground cumin
  2 tsp chili powder
  1 tsp paprika
  1 \nicefrac{1}{2} tsp crushed red pepper
  1 (10 oz) can tomato sauce
  2 large cans red kidney beans
  2 large cans black beans
  2 cans whole kernel corn, drained
\end{ingreds}

\begin{method}
  Saut\'e ground beef, onion, and garlic in butter.
  Add other ingredients and cook slowly for at least 2 hours.
\end{method}

\begin{comment}		%
\recipe[]{Red Beans and Rice*}
\serves{}
\freezerfriendly
\preptime{}
\cooktime{}
\begin{ingreds}
  
\end{ingreds}

\begin{method}
  
\end{method}

\end{comment}		%

\recipe[Beef jerky with a hint of root beer and chipotle.]{Root Beer Jerky}

\begin{ingreds}
  3 cups root beer
  \nicefrac{1}{3} cup soy sauce
  1 tsp chipotle powder
  1 tsp paprika
  1 tsp granulated garlic
  \nicefrac{1}{4} tsp liquid hickory smoke
  1 lb London broil strips, or other lean beef strips
\end{ingreds}

\begin{method}
  Mix all ingredients except the meat into a saucepan
  and heat over medium-high heat.
  Cook until the mixture reduces by half, 15 to 20 minutes.
  Remove from the heat and let cool completely.

  Pour the marinade into a 1-gallon resealable plastic
  freezer bag and add the meat strips to the marinade.
  Mix the strips around so they get completely coated with the marinade.

  Remove as much air as possible from the bag,
  seal, and place it in the refrigerator for 4 hours.
  During the marinating time, remove the bag
  from the refrigerator and work the meat around
  so the marinade is fully incorporated into it.

  Remove the strips from the marinade and arrrange
  in a single layer in a dehydrator.
  Dry at \ftemp{158} for 4-6 hours, or until done
  (jerky is dark and bends without snapping).
\end{method}

\recipe[Use pre-cut stew meat to save prep time on this Asian-inspired dish.]{Slow Cooker Beef and Broccoli}
\serves{4-6}
%\preptime{}
\cooktime{4.5 hours}
\begin{ingreds}
  1 lb beef stew meat
  1 cup beef broth
  \nicefrac{1}{2} cup soy sauce
  1 tbsp sesame oil
  4 cloves garlic, minced
  \nicefrac{1}{4} cup brown sugar
  1 large head of broccoli, cut into pieces
  \nicefrac{1}{4} cup water
  \nicefrac{1}{4} cup cornstarch
\end{ingreds}

\begin{method}
  Combine broth, soy sauce, sesame oil, garlic,
  and brown sugar in a slow cooker.
  Mix well and cook on low for 4 hours.

  Mix cornstarch and water in a small bowl.
  Add to slow cooker with broccoli pieces, stirring.
  Cook on low for another 30 minutes.

  Serve over rice.
\end{method}

%%
\chapter{Chicken and Turkey Mains}

\recipe[]{Mushroom Chicken Thighs*}
\serves{4}
\preptime{10 minutes}
\cooktime{45 minutes}
\begin{ingreds}
  3 tbsp olive oil
  4 cups sliced wild mushrooms
  2 tsp minced garlic
  1 tsp fresh thyme, chopped
  4 (6-oz) bone-in chicken thighs
  salt and pepper
\end{ingreds}
\begin{method}[Preheat oven to \ftemp{425}.]
  Heat 2 tbsp olive oil in a large ovenproof skillet
  over medium-high heat.
  
  Add the mushrooms and saut\'e for 10 minutes,
  until golden brown and lightly caramelized.
  
  Add the garlic and thyme and saut\'e 2 minutes more.
  
  Return the skillet to heat 
  and add the remaining olive oil.
  
  Season the chicken thighs with salt and pepper.
  
  Pan sear the chicken until lightly browned
  on both sides, about 10 minutes.
  
  Place the skillet into the oven and roast
  until the chicken is cooked through, 20 to 25 minutes. 
  
  Serve the chicken with a generous spoonful of 
  saut\'eed mushrooms.
\end{method}

\recipe[Served atop jasmine rice and green beans.]{Sweet-As-Honey Chicken}
\serves{2}
\preptime{5 minutes}
\cooktime{30 minutes}
\begin{ingreds}
  1 tbsp vegetable oil
  1 tsp minced garlic
  1 lime, cut into wedges
  1 thumb ginger, peeled and minced
  \nicefrac{1}{2} cup jasmine rice
  2 chicken breasts, 6 oz each
  1 \nicefrac{1}{2} tbsp white wine vinegar
  1 tbsp soy sauce
  1 pouch chicken stock concentrate
  1 oz honey
  6 oz green beans
  salt and pepper
\end{ingreds}
\begin{method}
  Bring 1 cup water and a pinch of salt
  to a boil in a small pot.
  Once water boils, add rice to pot.
  Cover, lower heat, and reduce to a gentle simmer.
  Cook until tender, about 15 minutes.

  Meanwhile, heat a drizzle of oil in a large pan
  over medium-high heat.
  Season chicken all over with salt and pepper.
  Add to pan and cook until no longer pink
  in the center, about 4-5 minutes per side.
  Remove from pan and set aside.

  Reduce heat under pan to medium-low
  and add a drizzle of oil.
  Saut\'e minced ginger and garlic until
  softened and fragrant, about 1 minute.
  Pour in vinegar and let reduce
  until almost dry and evaporated.

  Stir soy sauce, 2 tbsp water, stock concentrate,
  and honey into same pan.
  Let simmer until mixture thickens to
  a glaze-like consistency, 2-3 minutes.
  Remove from heat, then return chicken to pan
  and toss to coat.
  Set aside.

  Place green beans in a medium, microwave-safe bowl
  with a splash of water.
  Cover bowl with plastic wrap and poke
  a few holes in wrap.
  Microwave on high until just tender, about 2 minutes.
  Meanwhile, heat a drizzle of oil in a medium pan
  over medium-high heat.
  Thoroughly drain green beans, then add to pan.
  Cook, tossing occasionally, until lightly crisped, 3-5 minutes.
  Season with salt, pepper, and a squeeze of lime.

  Divide rice between plates;
  top with green beans and chicken.
  Serve with remaining lime wedges on the side for squeezing over.
\end{method}



\recipe[Tasty vegetable-filled turkey burgers.]{Tender Turkey Burgers}
\freezerfriendly
\serves{4}
\cooktime[Bake time]{30 minutes}
\preptime{15 minutes}
\begin{ingreds}
  2 tsp olive oil
  \nicefrac{1}{2} sweet onion, chopped
  \nicefrac{1}{2} red bell pepper, chopped
  1 celery stalk, finely chopped
  2 tsp minced garlic
  1 \nicefrac{1}{2} lb ground turkey
  dash sea salt
  dash freshly ground black pepper
  \columnbreak
  Olive oil spray
  4 sprouted grain buns
  Boston lettuce leaves, for garnish (optional)
  Red onion, thinly sliced, for garnish (optional)
  Tomato, sliced, for garnish (optional)
\end{ingreds}

\begin{method}[Preheat the oven to \ftemp{400}.]
  Heat the olive oil in a medium skillet over medium-high heat.
  Add the onion, red bell pepper, celery, and garlic
  and saut\'e until softened, about 4 minutes.

  Transfer the cooked vegetables to a medium bowl
  and add the turkey, salt, and pepper.
  Mix the ingredients together and divide the
  turkey mixture into four portions.

  Shape the portions into four patties about
  \nicefrac{3}{4} inch thick.
  Line a baking dish with parchment and place
  the turkey burgers on the dish in one layer.
  Lightly spray the patties with olive oil.

  Bake the burgers until they are golden and
  cooked through, about 25 minutes.
  Serve the burgers on buns topped with lettuce,
  onion, and tomato.
\end{method}

\begin{tips}
  Patties can be kept in the freezer --
  just put some parchment paper between them
  before wrapping them up.

  \medskip
  Also, don't be afraid to spread some of the veggies
  around the baking tray before putting it into the oven.
\end{tips}

%%
\chapter{Soups and Pasta Sauces}
\begin{comment}
Pasta is cheap, and soups can be prepared
in rather large batches --
perfect when you're on a college student budget (we are).
\end{comment}

\recipe[This recipe takes a while, so start it in the morning and leave it to cook all day.]{Bean Soup}
\serves{6}
\freezerfriendly
\cooktime{8 hours}
\preptime{20 minutes}
\begin{ingreds}
  1 bag 15 bean soup mix
  1 (32 oz) box chicken broth
  water
  1 ham bone with meat
  1 package Conecuh sausage
  1 onion, chopped
\end{ingreds}

\begin{method}
  Saut\'e sausage and onion until sausage
  is cooked and onion is soft.

  Place ham bone, sausage, and beans into slow cooker.
  Cover with chicken broth and water.
  Add half of the seasoning package from the soup mix bag and mix well.

  Cook on low for 8 hours.
\end{method}

\recipe[Easily improvised, this recipe serves as a nice base upon which to build.]{Chicken Soup}
\serves{}
\preptime{}
\cooktime{}
\freezerfriendly
\begin{ingreds}
  1 lb chicken breasts, cooked (may be shredded or chopped)
  1 tbsp butter
  2 stalks celery, chopped
  1 cup carrots, shredded
  1 package egg noodles
  4 (14.5 oz) cans chicken broth
  1 (14.5 oz) can vegetable broth
  \nicefrac{1}{2} tsp basil
  \nicefrac{1}{2} tsp oregano
  salt and pepper to taste
\end{ingreds}

\begin{method}
  In a large pot over medium heat, melt butter.
  Cook vegetables in butter until just tender, about 5 minutes.

  Pour in broth and stir in other ingredients.
  Boil, then reduce heat and simmer for 20 minutes before serving.
\end{method}

\recipe[Thick and creamy, this clam chowder pairs well with cold days.]{Clam Chowder}
\serves{6-8}
\preptime{15 minutes}
\cooktime{35 minutes}
\freezerfriendly
\begin{ingreds}
  \nicefrac{1}{2} lb thick cut bacon
  6 small cans chopped clams
  4 bottles clam juice
  6-8 red potatoes, diced
  2 cups heavy cream or half and half
  3 tbsp cornstarch
  1 stalk fresh rosemary, chopped finely
  1 medium onion, diced
  salt, pepper, garlic to taste
\end{ingreds}

\begin{method}
  In a large saucepan, boil potatoes until soft.

  Cut bacon into \nicefrac{1}{2} inch pieces,
  place in large stockpot, and fry until crispy.
  Remove bacon from pot onto paper towel.
  Drain grease from pot, leaving enough to layer the bottom.

  Cook onion in stockpot until soft.
  Add clams, clam juice, and spices into stock pot;
  cook on medium-high heat until warm.

  Pour heavy cream into lidded cup and add cornstarch.
  Shake well, then add to stockpot.
  Stir until liquid begins to thicken.

  Add potatoes to stockpot and simmer about 10-20 minutes.
  Add bacon to chowder or use as a garnish when serving.
\end{method}

\recipe[Basic spaghetti sauce. A great way to put some extra veggies on a plate.]{Spaghetti Sauce}
%\serves{}
\freezerfriendly
%\preptime{}
%\cooktime{}
\begin{ingreds}
  1 lb ground beef
  1 lb ground Italian sausage
  2 cups carrots, shredded
  1 zucchini, shredded
  1 yellow squash, shredded
  1 can olives, chopped
  2 tsp garlic, minced
  1 package mushrooms, finely chopped
  \columnbreak
  1 large onion, chopped
  3 (14 oz) cans pasta sauce
  Italian seasoning
  crushed red pepper
  chili powder
  cajun seasoning
  salt and pepper to taste
\end{ingreds}

\begin{method}
  In a large pot, brown meat with onion and garlic; drain.
  Return to heat and add carrots, zucchini, squash,
  mushrooms, and olives, cooking until vegetables are tender.

  Stir in pasta sauce one can at a time; season to taste.
  Reduce heat to low and simmer until uniformly heated,
  then remove from heat.

  Serve over noodles.
\end{method}
\begin{tips}
  If you're feeling adventurous, try serving
  over spaghetti squash instead of pasta.
  % anything special to do to it first?
\end{tips}


\chapter{Desserts and Drinks}
\begin{comment}
Meal planning isn't easy, so don't forget
to reward yourself with one (or more) of these desserts
and refreshing drinks.
\end{comment}

\recipe[Very easy to make -- you may even have all of the ingredients already.]{Banana Bread}
\vegetarian
%\serves{}
\preptime{15 minutes}
\cooktime[Bake time]{45 minutes}
\begin{ingreds}
  4 very ripe bananas, peeled
  \nicefrac{1}{3} cup melted butter
  1 tsp baking soda
  pinch of salt
  \nicefrac{3}{4} cup sugar
  1 large egg, beaten
  1 tsp vanilla extract
  1 \nicefrac{1}{2} cups all-purpose flour
  1 tsp cinnamon
  crushed pecans
\end{ingreds}

\begin{method}[Preheat oven to \ftemp{350} and butter the baking pan.]
  In a mixing bowl, mash the ripe bananas with
  a fork until completely smooth.
  Stir the melted butter into the mashed bananas.

  Mix in the baking soda and salt.
  Stir in the sugar, beaten egg, vanilla extract, and cinnamon.
  Mix in the flour and any additions.

  Pour the batter into the prepared pan,
  and sprinkle crushed pecans atop the batter.
  Bake for 45 minutes, or until a tester inserted
  into the center comes out clean.

  Remove from the oven and cool completely on a rack.
  Remove the banana bread from the pan and serve.
\end{method}
\begin{tips}
  Rather than chopping pecans manually (it'll take a while),
  wrap them in a paper towel and use the side of
  a flat knife to crush them.
\end{tips}

\recipe[An easy way to make cookies out of any cake mix.]{Cakies}
\vegetarian
%\preptime{}
\cooktime[Bake time]{10 minutes}
\begin{ingreds}
  1 box cake mix
  2 eggs
  \nicefrac{1}{2} cup vegetable oil
  nuts or fruit (optional)
\end{ingreds}
\begin{method}[Preheat oven to \ftemp{375}.]
  Mix all ingredients together and roll into walnut-sized balls.
  
  Place 2 inches apart on ungreased cookie sheet.
  
  Bake 8--10 minutes. 
  Remove and cool for 2 minutes.
\end{method}

\recipe[Can be made with regular or almond milk.]{Chai Tea Latte*}
\serves{1}
\begin{ingreds}
  2 chai tea bags
  \nicefrac{1}{4} cup boiling water
  \nicefrac{3}{4} cup milk
  honey or sugar to taste (optional)
  cinnamon or nutmeg, for garnish (optional)
\end{ingreds}
\begin{method}
  Steep the tea bags in the hot water
  for 3 to 5 minutes.
  
  Warm milk in a pan, then whisk with a frother. 
  Combine the milk and tea, 
  adding honey or sugar as desired. 
  Sprinkle with cinnamon or nutmeg and serve.
\end{method}

\recipe[]{Creamy Pecan Pralines}
\serves{18 pcs.}
\begin{ingreds}
  1 cup granulated sugar
  1 cup (packed) light brown sugar
  2 tablespoons light corn syrup
  2 tablespoons butter
  Pinch of salt
  1/2 cup sweetened condensed milk
  1 teaspoon vanilla extract
  1 1/2 cups pecan pieces
\end{ingreds}
\begin{method}
  Combine the sugar, brown sugar, corn syrup, butter, salt,
  and condensed milk in a heavy saucepan over medium heat.

  With a wooden spoon, stir until the sugar dissolves.
  Continue to cook, stirring, until smooth and light brown,
  about 8 minutes.

  Add the vanilla and pecans and continue to cook,
  stirring, until the mixture reaches \ftemp{234} to
  \ftemp{240} on a candy thermometer or the soft ball stage,
  that is, when a bit dropped into cold water
  forms a soft ball that flattens.

  Remove from the heat and drop by the spoonful onto wax paper.
  Let cool.

  Remove from the paper with a thin knife.
  Pralines can be stored in an airtight container
  at room temperature for up to 2 weeks.
\end{method}
\begin{tips}
  If you attempt to roast the pecans yourself,
  be sure to check them frequently!
  They can and \emph{will} catch fire if left unattended
  (we know from experience).
\end{tips}

\recipe[]{Easy Penuche*}
\preptime{5 minutes}
\cooktime{10 minutes}
\serves{90 pcs.}
\begin{ingreds}
  1 cup unsalted butter
  2 cups brown sugar, packed
  \nicefrac{1}{2} cup whole milk
  1 tsp vanilla extract
  3 \nicefrac{1}{2} to 4 cups sifted powdered sugar
  1 \nicefrac{1}{2} cups coarsely chopped pecans
\end{ingreds}
\begin{method}
  Spray a 9x5" loaf pan with cooking spray
  and line with parchment paper.
  Set aside.

  Melt butter in a heavy-bottomed 2-quart saucepan
  over \linebreak medium-low heat, stirring as needed.
  Stir in the brown sugar and cook for
  2 minutes, stirring constantly.

  Gently pour in the milk and, stirring constantly,
  bring mixture to a boil.
  Remove from heat and let cool for 30-40 minutes,
  or until mixture is lukewarm or room temperature.

  Stir in the vanilla extract.
  Stir in powdered sugar, 1 to \nicefrac{1}{2} cup
  at a time, stirring until completely combined.
  Add powdered sugar just until the mixture is
  thick and has the consistency of fudge.

  Stir in pecans, then spoon into the prepared
  loaf pan and top with additional pecans if desired.
  Cover and cool until firm.

  Cut into squares and store in the refrigerator
  in an airtight container.
\end{method}

\recipe[Add an equal amount of unsweetened black tea for an extra refreshing drink.]{Lemonade}
%\serves{}
%\preptime{}
%\cooktime{}
\vegan
\begin{ingreds}
  1 \nicefrac{3}{4} cups white sugar
  8 cups water
  1 \nicefrac{1}{2} cups lemon juice
\end{ingreds}

\begin{method}
  In a small saucepan, combine sugar and 1 cup water.
  Bring to a boil and stir to dissolve sugar.
  Allow to cool to room temperature, then cover
  and refrigerate until chilled.

  Remove seeds from lemon juice, leaving pulp (if applicable).
  In pitcher, stir together chilled syrup, lemon juice,
  and remaining 7 cups water.
\end{method}
\begin{tips}
  If you don't want (or have time) to wait
  for the syrup to chill slowly,
  add ice to it instead.
\end{tips}

\recipe[]{Microwave Mug Brownie*}
\serves{1}
\preptime{6 minutes}
\cooktime{1 minutes}
\begin{ingreds}
  2 tbsp melted butter or oil
  2 tbsp water, milk, or chocolate milk
  \nicefrac{1}{4} tsp vanilla extract
  dash salt
  2 tbsp white sugar
  2 tbsp unsweetened cocoa powder
  4 tbsp flour
\end{ingreds}
\begin{method}
  In a coffee mug, add water/milk
  melted butter/oil, a dash of salt, 
  and vanilla extract. Whisk well.

  Add cocoa powder, sugar, and flour,
  whisking well after each.

  Microwave for 60 seconds.
  Center should be slightly molten.
  Do not overcook.

  Enjoy with a spoon. 
\end{method}

\recipe[Can be used as a fruit dip or eaten by itself.]{Sweet Potato Dip}
\serves{2 \nicefrac{1}{2} cups}
\preptime{10 minutes}
\cooktime{30 minutes}
\freezerfriendly
\vegetarian%?
\begin{ingreds}
  2 cups sweet potato, cooked and pur\'eed
  \nicefrac{1}{2} cup vanilla yogurt
  1 tbsp raw honey
  1 tsp pure vanilla extract
  \nicefrac{1}{4} tsp ground cinnamon
  \nicefrac{1}{8} tsp ground nutmeg
  \nicefrac{1}{8} tsp ground ginger
  dash ground allspice
  fruit, for dipping (optional)
\end{ingreds}
\begin{method}
  In a medium bowl, whisk together 
  the sweet potato, honey, vanilla, 
  and seasonings until well blended.
  
  Serve with cut up fruit, if desired.
  
  Store any extra in a sealed container.
\end{method}
\begin{tips}
  If you don't want to cook the potatoes yourself, 
  you can use two thawed frozen sweet potato patties.
\end{tips}

\begin{comment} % NOTE this is an example recipe that came with the format
\recipe[This is a simple headnote that describes the product for the user. A simple but elegant dessert.]{Raspberry Chocolate Tiramisu (example)}
\serves{4}
\preptime{1 hour}
\cooktime[Chill time]{1 \nicefrac{1}{2} hours}
\vegetarian
\freezerfriendly
\begin{ingreds}
  100ml Double Strength Coffee
  400g Raspberries (blitzed)
  200g Mascarpone
  2 tbsp Sweetener
  1 tsp Vanilla Extract
  700g Vanilla Yogurt
  15g Dark Chocolate (finely grated) and a really long one
\columnbreak
\ingredients[For the Crumble Mixture:]
  80g Wholemeal Flour
  80g Plain Flour
  80g Butter (diced)
  70g Demerara Sugar
\end{ingreds}

\begin{method}[Preheat the oven to Gas Mark 4, Electric \ctemp{180}, Fan \ctemp{160}.]
  Stir the two kinds of flour together in a bowl,
  add the butter and rub it into the flour.
  When the mixture looks like breadcrumbs, mix in the brown sugar.
  Lay the mixture on a shallow baking tray and bake for
  25--30 minutes until golden brown.
  Leave on the side to cool.

  Mix together the mascarpone, sweetener, vanilla extract,
  and three quarters of the chocolate.
  Put half the crumble mixture in each of the glasses
  and pour over half the quark mixture along with
  half the raspberries.

  Put the other half of the crumble mixture on top,
  followed by the remaining quark mixture and raspberries.
  Sprinkle over the last of the chocolate.
  Chill for 3 hours before serving.
\end {method}
\showit[1.25in]{example-image-a}{This is a picture}
\end{comment}

%\backmatter
%\chapter{Acknowledgements}


\newpage
\phantomsection
\addcontentsline{toc}{section}{\indexname}
\printindex

\end{document}
