\chapter{Meat-free Mains}
\begin{comment} % TODO finish chapter intro
We love meats, but meats are expensive,
sometimes making up as much as half the total
cost of our weekly grocery trip.
Vegetarian options can be as much about practicality
as about lifestyle choices.
\end{comment}

\recipe[So tasty you won't even notice it has no meat.]{Black Beans and Quinoa}
\vegan
\serves{4}
\preptime{10 minutes}
\cooktime{30 minutes}
\begin{ingreds}
  1 tbsp olive oil
  1 medium sweet or yellow onion, diced
  2 cloves garlic, minced
  \nicefrac{3}{4} cup uncooked quinoa, rinsed
  1 (15 oz) can black beans, drained and rinsed
  1 tsp chili powder
  1 tsp cumin
  \nicefrac{1}{4} tsp crushed red pepper flakes
  \nicefrac{1}{2} tsp black pepper
  salt to taste
  1 (10 oz) can original Rotel
  \nicefrac{1}{2} cup fresh cilantro, chopped
  1 \nicefrac{3}{4} cup vegetable broth
\end{ingreds}

\begin{method}
  In a large skillet, heat olive oil on medium-low heat
  and saut\'e diced onions until tender, about 4 minutes.
  Add garlic and saut\'e one additional minute.
  Add the remaining ingredients in the order listed above.

  Cover and bring to a boil.
  Reduce heat to a low boil and cook 15-20 minutes
  or until liquid is absorbed.
  Remove from heat and allow to sit covered for 5 minutes before serving.

  Fluff quinoa with a large spoon and serve.
\end{method}
\begin{tips}
  Don't forget to rinse the quinoa!
  Dry quinoa has a bitter coating that needs
  to be removed before cooking.
\end{tips}
